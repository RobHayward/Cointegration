\documentclass{article}\usepackage{graphicx, color}
%% maxwidth is the original width if it is less than linewidth
%% otherwise use linewidth (to make sure the graphics do not exceed the margin)
\makeatletter
\def\maxwidth{ %
  \ifdim\Gin@nat@width>\linewidth
    \linewidth
  \else
    \Gin@nat@width
  \fi
}
\makeatother

\definecolor{fgcolor}{rgb}{0.2, 0.2, 0.2}
\newcommand{\hlnumber}[1]{\textcolor[rgb]{0,0,0}{#1}}%
\newcommand{\hlfunctioncall}[1]{\textcolor[rgb]{0.501960784313725,0,0.329411764705882}{\textbf{#1}}}%
\newcommand{\hlstring}[1]{\textcolor[rgb]{0.6,0.6,1}{#1}}%
\newcommand{\hlkeyword}[1]{\textcolor[rgb]{0,0,0}{\textbf{#1}}}%
\newcommand{\hlargument}[1]{\textcolor[rgb]{0.690196078431373,0.250980392156863,0.0196078431372549}{#1}}%
\newcommand{\hlcomment}[1]{\textcolor[rgb]{0.180392156862745,0.6,0.341176470588235}{#1}}%
\newcommand{\hlroxygencomment}[1]{\textcolor[rgb]{0.43921568627451,0.47843137254902,0.701960784313725}{#1}}%
\newcommand{\hlformalargs}[1]{\textcolor[rgb]{0.690196078431373,0.250980392156863,0.0196078431372549}{#1}}%
\newcommand{\hleqformalargs}[1]{\textcolor[rgb]{0.690196078431373,0.250980392156863,0.0196078431372549}{#1}}%
\newcommand{\hlassignement}[1]{\textcolor[rgb]{0,0,0}{\textbf{#1}}}%
\newcommand{\hlpackage}[1]{\textcolor[rgb]{0.588235294117647,0.709803921568627,0.145098039215686}{#1}}%
\newcommand{\hlslot}[1]{\textit{#1}}%
\newcommand{\hlsymbol}[1]{\textcolor[rgb]{0,0,0}{#1}}%
\newcommand{\hlprompt}[1]{\textcolor[rgb]{0.2,0.2,0.2}{#1}}%

\usepackage{framed}
\makeatletter
\newenvironment{kframe}{%
 \def\at@end@of@kframe{}%
 \ifinner\ifhmode%
  \def\at@end@of@kframe{\end{minipage}}%
  \begin{minipage}{\columnwidth}%
 \fi\fi%
 \def\FrameCommand##1{\hskip\@totalleftmargin \hskip-\fboxsep
 \colorbox{shadecolor}{##1}\hskip-\fboxsep
     % There is no \\@totalrightmargin, so:
     \hskip-\linewidth \hskip-\@totalleftmargin \hskip\columnwidth}%
 \MakeFramed {\advance\hsize-\width
   \@totalleftmargin\z@ \linewidth\hsize
   \@setminipage}}%
 {\par\unskip\endMakeFramed%
 \at@end@of@kframe}
\makeatother

\definecolor{shadecolor}{rgb}{.97, .97, .97}
\definecolor{messagecolor}{rgb}{0, 0, 0}
\definecolor{warningcolor}{rgb}{1, 0, 1}
\definecolor{errorcolor}{rgb}{1, 0, 0}
\newenvironment{knitrout}{}{} % an empty environment to be redefined in TeX

\usepackage{alltt}
\usepackage{amsmath}
\usepackage{natbib}
\title{Cointegration}
\author{Rob Hayward}
\IfFileExists{upquote.sty}{\usepackage{upquote}}{}
\begin{document}
\maketitle
\section{Introduction}
This paper will examine time series with a focus on stationarity, integration and cointegration. The first part looks at univariate series, the second looks at multivariate and the final section assesses conintegration.  The final section concludes.  

\section{Stationary data}
The standard series to be investigated can take the form of 
\begin{equation}
y_t = TD_t + z_t
\end{equation}

It is possible to differentiate between \emph{trend stationary} and \emph{difference stationary} processes.  

\begin{equation}
y_t = y_{t-1} + \mu = y_0 + \mu t
\end{equation}
and
\begin{equation}
y_t = y_{t-1} + \varepsilon = y_0 + \sum_{i=0}^t \varepsilon_t
\end{equation}


If all the roots of the autoregressive polynominal $\phi_p(z)$ lie outside the unit circle, there is a trend stationary process; if at least one of the roots lies on the unit circule and there is a difference statonary process. 

\begin{equation}
\phi_p(z) = 1 - \phi_1 (z) - \phi_2(z)^2 - \phi_3(z)^3..\phi_p(z)^p
\end{equation}
 
It is possible to create and plot these different types of time series. 
\begin{knitrout}
\definecolor{shadecolor}{rgb}{0.969, 0.969, 0.969}\color{fgcolor}\begin{kframe}
\begin{alltt}
\hlfunctioncall{set.seed}(123456)
e <- \hlfunctioncall{rnorm}(500)
rw.nd <- \hlfunctioncall{cumsum}(e)
\end{alltt}
\end{kframe}
\end{knitrout}

\begin{knitrout}
\definecolor{shadecolor}{rgb}{0.969, 0.969, 0.969}\color{fgcolor}\begin{kframe}
\begin{alltt}
trd <- 1:500
\hlcomment{# random walk with drift}
rw.wd <- 0.5 * trd + \hlfunctioncall{cumsum}(e)
\hlcomment{# deterministic trend and noise}
dt <- e + 0.5 * trd
\end{alltt}
\end{kframe}
\end{knitrout}

\begin{knitrout}
\definecolor{shadecolor}{rgb}{0.969, 0.969, 0.969}\color{fgcolor}\begin{kframe}
\begin{alltt}
\hlfunctioncall{par}(mar = \hlfunctioncall{rep}(5, 4))
\hlfunctioncall{plot.ts}(dt, lty = 1, ylab = \hlstring{""}, xlab = \hlstring{""})
\hlfunctioncall{lines}(rw.wd, lty = 2)
\hlfunctioncall{par}(new = T)
\hlfunctioncall{plot.ts}(rw.nd, lty = 3, axes = FALSE)
\hlfunctioncall{axis}(4, \hlfunctioncall{pretty}(\hlfunctioncall{range}(rw.nd)))
\hlfunctioncall{lines}(rw.nd, lty = 3)
\hlfunctioncall{legend}(10, 18.7, legend = \hlfunctioncall{c}(\hlstring{"det. trend + \hlfunctioncall{noise} (ls)"}, \hlstring{"rw \hlfunctioncall{drift} (ls)"}, \hlstring{"\hlfunctioncall{rw} (rs)"}), 
    lty = \hlfunctioncall{c}(1, 2, 3))
\end{alltt}
\end{kframe}
\includegraphics[width=\maxwidth]{figure/plotting} 

\end{knitrout}

There are also a series of tests that can be put in place to assess the nature of the time series.  There are three types of stationary series: \emph{trend stationary}, \emph{difference stationary} and \emph{difference stationary with a constant}

Rather than estimating  
\begin{equation}
y_t = \beta_1 + \beta_2 t + \rho y_{t-1} +\sum_{j=1}^k \gamma_j \Delta y_i + u_{1t}
\end{equation}
to test whether $\rho$ is equal to unity, $y_{t-1}$ is taken from each side to produce. 

\begin{equation}
\Delta y_t = \beta_1 + \beta_2 t + \pi y_{t-1} +\sum_{j=1}^k \gamma_j \Delta y_i + u_{1t}
\label{eq:df}
\end{equation}

where $\pi$ is equal to $1 - \rho$ and therefore if $\pi$ is significantly different from zero, $\rho$ cannot be one and there is no unit root. 

Lags of the dependent variable are used to remove any serial correlation in the residuals. \emph{Information Criteria} and t-statistics can be used to assess the appropriate number of lags

Using the usca package and the ur.df function on UK real consumer spending data (lc). Set up the data as a timeseries.   

\begin{knitrout}
\definecolor{shadecolor}{rgb}{0.969, 0.969, 0.969}\color{fgcolor}\begin{kframe}
\begin{alltt}
\hlfunctioncall{library}(urca)
\hlfunctioncall{library}(xtable)
\hlfunctioncall{data}(Raotbl3)
lc <- \hlfunctioncall{ts}(Raotbl3$lc, start = \hlfunctioncall{c}(1966, 4), end = \hlfunctioncall{c}(1991, 2), frequency = 4)
\end{alltt}
\end{kframe}
\end{knitrout}

Test the trend (lc.ct), drift (lc.co) and first difference (lc2.ct). 
\begin{knitrout}
\definecolor{shadecolor}{rgb}{0.969, 0.969, 0.969}\color{fgcolor}\begin{kframe}
\begin{alltt}
lc.ct <- \hlfunctioncall{ur.df}(lc, lags = 3, type = \hlstring{"trend"})
lc.co <- \hlfunctioncall{ur.df}(lc, lags = 3, type = \hlstring{"drift"})
lc2 <- \hlfunctioncall{diff}(lc)
lc2.ct <- \hlfunctioncall{ur.df}(lc2, type = \hlstring{"trend"}, lags = 3)
\end{alltt}
\end{kframe}
\end{knitrout}

The three different equations are tested by 'trend', 'drift' or 'none' and there are two tests that take place. 

The first $(\tau_3)$ tests whether $\pi$ is equal to zero.  The test is the usual t-value on the lagged dependent variable. This can seen in the summary() function or the SlotName "teststat".  The critical values for the test statistics are in the slotName "cval".  The following code extracts the relevant values and puts them into a table. 
\begin{kframe}
\begin{alltt}
a <- \hlfunctioncall{cbind}(\hlfunctioncall{t}(lc.ct@teststat), lc.ct@cval)
\hlfunctioncall{print}(\hlfunctioncall{xtable}(a, digits = 2))
\end{alltt}
\end{kframe}% latex table generated in R 2.15.2 by xtable 1.7-0 package
% Sat Jun 08 17:34:40 2013
\begin{table}[ht]
\begin{center}
\begin{tabular}{rrrrr}
  \hline
 & statistic & 1pct & 5pct & 10pct \\ 
  \hline
tau3 & -2.24 & -4.04 & -3.45 & -3.15 \\ 
  phi2 & 3.74 & 6.50 & 4.88 & 4.16 \\ 
  phi3 & 2.60 & 8.73 & 6.49 & 5.47 \\ 
   \hline
\end{tabular}
\end{center}
\end{table}



The $\tau_3$ test statistic is the test of the null hypothesis that the coefficient on the difference of the lagged dependent variable is equal to zero and that there is a \emph{unit root} as $\rho$ is equal to one.  


The critical value for a sample size of 100 comes from \citep{Fuller1976}. 

An F-test of the null hypothesis that the coefficients on the lagged change in the dependend variable and the coefficient on the time trend are jointly equal to zero is also supplied $(\phi_3)$.  The critical values come from Table VI \citep{DF1981} testing the null $(\alpha, \beta, \rho) = (\alpha, 0, 1)$.  It seems that unit root and lack of time trend cannot be rejected. A joint test of the null that the coefficients on the drift, time trend and lagged difference of the dependent variable is suppoed in $(\phi_2)$.  The critical values come from Table V \citep{DF1981} testing the null $(\alpha, \beta, \rho) = (0, 0, 1)$.

There are a number of other tests of a unit root in the Bernhard Pfaff text.  However, these all test the null of a unit root.  The \emph{Kwiatkowski-Phillips-Schmidt-Shin Test} \citep{KPSS} tests the null stationarity.  This is a much more powerful test and can be used in conjunction with the more conventional tests.  If the other tests suggest a unit root but the KPSS rejects a unit root, it is probably best to consider the data as stationary. 

The KPSS test is of the form
\begin{equation}
\label{eq:KPSS}
y_t = \zeta t + r_t + \varepsilon_t
\end{equation}

\begin{equation}
r_t = r_{t-1} + u_t
\end{equation}

The test statistic is calculated by running the regression of $y$ on a constant and trend as in equation \ref{eq:KPSS} or on just a constant as in equation \ref{eq:KPSS} with $\zeta$ equal to zero.    

\begin{equation}
LM = \frac{\sum_{i = 1}^T S_t^2}{\hat{\sigma_t^2}}
\end{equation}
where 
\begin{equation}
S_t = \sum_{i = 1}^t \hat{\varepsilon} , t = 1, 2,...T
\end{equation}
and the estimate of the error variance
\begin{equation}
\hat{\sigma_{\varepsilon}}^2 = s^2 (l) = T^{-1} \sum{t=1}^T +2T - 1 \sum{s=1}^l 1-\frac{s}{l+1} \sum_{t=s+1}^T \hat{\varepsilon} \hat{\varepsilon_{t-1}}
\end{equation}

\section{Cointegration}
This is the overview of conintegration and the methods use to analyse conintegrated relationships. Non-stationary data may exhibit \emph{spurious regression}.  If two norman random variables are created (e1 and e2) and two series (y1 and y2) have a trend plus a random shock. 
\begin{knitrout}
\definecolor{shadecolor}{rgb}{0.969, 0.969, 0.969}\color{fgcolor}\begin{kframe}
\begin{alltt}
\hlfunctioncall{library}(lmtest)
\hlfunctioncall{library}(xtable)
\hlfunctioncall{set.seed}(123456)
e1 <- \hlfunctioncall{rnorm}(500)
e2 <- \hlfunctioncall{rnorm}(500)
trd <- 1:500
y1 <- 0.8 * trd + \hlfunctioncall{cumsum}(e1)
y2 <- 0.6 * trd + \hlfunctioncall{cumsum}(e2)
\end{alltt}
\end{kframe}
\end{knitrout}

Now plot the two series
\begin{knitrout}
\definecolor{shadecolor}{rgb}{0.969, 0.969, 0.969}\color{fgcolor}\begin{kframe}
\begin{alltt}
\hlfunctioncall{plot}(y1, type = \hlstring{"l"}, main = \hlstring{"Plot of y1 and y2"}, col = \hlstring{"red"}, ylab = \hlstring{"y1, y2"})
\hlfunctioncall{lines}(y2, col = \hlstring{"blue"})
\end{alltt}
\end{kframe}
\includegraphics[width=\maxwidth]{figure/plot} 

\end{knitrout}

Run a regression of $y1$ on $y2$ and it appears that there is a strong relationship.  
\begin{kframe}
\begin{alltt}
sr.reg <- \hlfunctioncall{lm}(y1 ~ y2)
\hlfunctioncall{print}(\hlfunctioncall{xtable}(sr.reg))
\end{alltt}
\end{kframe}% latex table generated in R 2.15.2 by xtable 1.7-0 package
% Sat Jun 08 17:34:40 2013
\begin{table}[ht]
\begin{center}
\begin{tabular}{rrrrr}
  \hline
 & Estimate & Std. Error & t value & Pr($>$$|$t$|$) \\ 
  \hline
(Intercept) & -29.3270 & 1.3672 & -21.45 & 0.0000 \\ 
  y2 & 1.4408 & 0.0075 & 191.62 & 0.0000 \\ 
   \hline
\end{tabular}
\end{center}
\end{table}


However, the Durbin-Watson statistics shows there is a large amount of auto-correlation in the residuals.  
\begin{knitrout}
\definecolor{shadecolor}{rgb}{0.969, 0.969, 0.969}\color{fgcolor}\begin{kframe}
\begin{alltt}
sr.dw <- \hlfunctioncall{dwtest}(sr.reg)$statistic
sr.dw
\end{alltt}
\begin{verbatim}
##      DW 
## 0.01715
\end{verbatim}
\end{kframe}
\end{knitrout}

The statistic will be around 2 if there is no autocorrelation. 
\newpage
\bibliography{myref}
\bibliographystyle{agsm}


\end{document}
